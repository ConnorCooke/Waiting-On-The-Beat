%%%%%%%%%%%%%%%%%%%%%%%%%%%%%%%%%%%%%%%%%
% Journal Article
% LaTeX Template
% Version 1.4 (15/5/16)
%
% This template has been downloaded from:
% http://www.LaTeXTemplates.com
%
% Original author:
% Frits Wenneker (http://www.howtotex.com) with extensive modifications by
% Vel (vel@LaTeXTemplates.com)
%
% License:
% CC BY-NC-SA 3.0 (http://creativecommons.org/licenses/by-nc-sa/3.0/)
%
%%%%%%%%%%%%%%%%%%%%%%%%%%%%%%%%%%%%%%%%%

%----------------------------------------------------------------------------------------
%	PACKAGES AND OTHER DOCUMENT CONFIGURATIONS
%----------------------------------------------------------------------------------------

\documentclass[twoside,twocolumn]{article}

\usepackage{blindtext} % Package to generate dummy text throughout this template 

\usepackage[sc]{mathpazo} % Use the Palatino font
\usepackage[T1]{fontenc} % Use 8-bit encoding that has 256 glyphs
\linespread{1.05} % Line spacing - Palatino needs more space between lines
\usepackage{microtype} % Slightly tweak font spacing for aesthetics

\usepackage[english]{babel} % Language hyphenation and typographical rules

\usepackage[hmarginratio=1:1,top=32mm,columnsep=20pt]{geometry} % Document margins
\usepackage[hang, small,labelfont=bf,up,textfont=it,up]{caption} % Custom captions under/above floats in tables or figures
\usepackage{booktabs} % Horizontal rules in tables

\usepackage{lettrine} % The lettrine is the first enlarged letter at the beginning of the text

\usepackage{enumitem} % Customized lists
\setlist[itemize]{noitemsep} % Make itemize lists more compact

\usepackage{abstract} % Allows abstract customization
\renewcommand{\abstractnamefont}{\normalfont\bfseries} % Set the "Abstract" text to bold
\renewcommand{\abstracttextfont}{\normalfont\small\itshape} % Set the abstract itself to small italic textd

\usepackage{titlesec} % Allows customization of titles
\renewcommand\thesection{\Roman{section}} % Roman numerals for the sections
\renewcommand\thesubsection{\roman{subsection}} % roman numerals for subsections
\titleformat{\section}[block]{\large\scshape\centering}{\thesection.}{1em}{} % Change the look of the section titles
\titleformat{\subsection}[block]{\large}{\thesubsection.}{1em}{} % Change the look of the section titles

\usepackage{fancyhdr} % Headers and footers
\pagestyle{fancy} % All pages have headers and footers
\fancyhead{} % Blank out the default header
\fancyfoot{} % Blank out the default footer
\fancyhead[C]{Running title $\bullet$ May 2016 $\bullet$ Vol. XXI, No. 1} % Custom header text
\fancyfoot[RO,LE]{\thepage} % Custom footer text

\usepackage{titling} % Customizing the title section

\usepackage{hyperref} % For hyperlinks in the PDF

%----------------------------------------------------------------------------------------
%	TITLE SECTION
%----------------------------------------------------------------------------------------

\setlength{\droptitle}{-4\baselineskip} % Move the title up

\pretitle{\begin{center}\Huge\bfseries} % Article title formatting
\posttitle{\end{center}} % Article title closing formatting
\title{Waiting on the Beat Playtesting} % Article title
\author{%
\textsc{Connor Cooke}
\and
\textsc{Sarah Piot}
\and
\textsc{Cortland Laidlaw}
\and
\textsc{Josh Stanzeleit}
\and
\textsc{Eileen Van Heerde}
\and
\textsc{Martin Thingvold}\\
%\normalsize University of Saskatchewan \\ % Your institution
%\normalsize \href{mailto:mat534@usask.ca}{mat534@usask.ca}
% Your email address
%\\
%\normalsize \href{mailto:evv446@usask.ca}{evv446@usask.ca} % Your email address
%\and % Uncomment if 2 authors are required, duplicate these 4 lines if more
%\textsc{Jane Smith}\thanks{Corresponding author} \\[1ex] % Second author's name
%\normalsize University of Utah \\ % Second author's institution
%\normalsize \href{mailto:jane@smith.com}{jane@smith.com} % Second author's email address
}
\date{\today} % Leave empty to omit a date
\renewcommand{\maketitlehookd}{%
%\begin{abstract}
%\noindent I don't know how to write an abstract.
%\noindent \blindtext % Dummy abstract text - replace \blindtext with your abstract text
%\end{abstract}
}

%----------------------------------------------------------------------------------------

\begin{document}

% Print the title
\maketitle

%----------------------------------------------------------------------------------------
%	ARTICLE CONTENTS
%----------------------------------------------------------------------------------------

\section{Introduction}

\lettrine[nindent=0em,lines=3]{T} he goal of playtesting is to have users interact
with our product and to utilize their feedback to improve the product. Our group chose
6%TODO change number of participants
participants to run through our alpha and give feedback. The playtesting had three components the evaluator followed; we first \hyperlink{section.2}{introduced the product to the player,} then we \hyperlink{section.3}{observed the player engaging with the
game and transcribing their comments,} and finally we \hyperlink{section.4}{had players answer specific questions.}




%------------------------------------------------

\section{Playtesting Script}
\label{sec:script}
To remove biases from our results it's important to follow a script to ensure that all players are introduced
to the product in the same way. The final script follows this principle while avoid "hand holding" with the player.
The way we compromised these two principles was by thinking about what information would be given to the player in a finished product. Our script consists of the following points:

\begin{itemize}
\item Our group has been working on a game called Waiting on the Beat
\item It's a rhythm game, so you have to hit the arrow keys to the beat of the music
\item There's a helpful visual component that should help you keep the beat at the bottom of the screen
\item Use the arrow keys to control the movement.
\item The goal is to take orders, get food from the bar, bring the food to the customer and collect money.
\item Here's where the bar is to submit orders and here is where you get food.
\end{itemize}

%Text requiring further explanation\footnote{Example footnote}.

%------------------------------------------------

\section{Observation}

From observing we received a bunch of criticism
%TODO WRITE THIS



%------------------------------------------------

\section{Post Gameplay Questions}
After the player was finished engaging with the game we asked several questions to
attempt to get more clear constructive feedback and locating issues with different
 users. The group decided to use Google Forms to Process feedback which allows the group
 to correlate similiarities between players with better game literacy and players who haven't played many video games. We decided to ask the following questions:
%TODO Discuss using google forms

\begin{itemize}
\item How much experience do you have playing games?(Linear scale)
\item How much experience do you have with rhythm games? (Linear scale)
\item Do you have any experience playing a musical instrument?(Linear scale)
\item How difficult did you find the game?(Linear scale)
\item Was there anything that you found frustrating?(long answer)
\item What changes would you suggest to improve the game? (Long answer)
\item How intuitive did the controls feel?(Long answer)
\item What did you like about the game?(Long answer)
\item What did you not like about the game?(Long answer)
\item What changes would you suggest we make to the game?(Long answer)
\end{itemize}


%----------------------------------------------------------------------------------------
%	REFERENCE LIST
%----------------------------------------------------------------------------------------

%\begin{thebibliography}{99} % Bibliography - this is intentionally simple in this template

%\bibitem[Figueredo and Wolf, 2009]{Figueredo:2009dg}
%Figueredo, A.~J. and Wolf, P. S.~A. (2009).
%\newblock Assortative pairing and life history strategy - a cross-cultural
%  study.
%\newblock {\em Human Nature}, 20:317--330.
 
%\end{thebibliography}

%----------------------------------------------------------------------------------------

\end{document}
